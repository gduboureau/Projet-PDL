\documentclass[a4paper,12pt]{article}
\usepackage[frenchb]{babel}
\usepackage[utf8]{inputenc}
\usepackage[T1]{fontenc} 
\usepackage{lmodern}
\usepackage{mathptmx}

\usepackage{amsmath}
\usepackage{etoolbox}
\usepackage{float}
\usepackage{geometry}
\usepackage{hyperref}
\hypersetup{
    colorlinks=true,
    linkcolor=blue,
    filecolor=magenta,      
    urlcolor=cyan,
    pdftitle={Overleaf Example},
    pdfpagemode=FullScreen,
    }
\usepackage{graphicx}
%\usepackage[disable]{todonotes}
\usepackage{todonotes}
\usepackage{titlesec}
\titleformat*{\section}{\Large\bfseries\sffamily}
\titleformat*{\subsection}{\large\bfseries\sffamily}
\titleformat*{\subsubsection}{\itshape\subsubsectionfont}

\geometry{margin=2cm}

\newcounter{besoin}

% Descriptif des besoins:
% 1 - Label du besoin pour référencement 
% 2 - Titre du besoin
% 3 - Description
% 4 - Gestion d'erreurs
% 5 - Spécifications tests
\newcommand{\besoin}[5]{%
  \refstepcounter{besoin}%
  \fbox{\parbox{0.95\linewidth}{%
    \begin{center}\label{besoin:#1}\textbf{\sffamily Besoin~\thebesoin~: #2}\end{center}
    \ifstrempty{#3}{}{\textbf{Description~:} #3\par}
    \vspace{0.5em}
    \ifstrempty{#4}{}{\textbf{Gestion d'erreurs~:} #4\par}
    \vspace{0.5em}
    \ifstrempty{#5}{}{\textbf{Tests~:} #5\par}
  }}
}

\newcommand{\refBesoin}[1]{%
  Besoin~\ref{besoin:#1}%
}

\title{\sffamily \textbf{Projet de Développement Logiciel}}
\author{L3 Informatique -- Université de Bordeaux}
\date{}

\begin{document}

\maketitle


\section{Introduction}

Ce document présente les besoins nécessaires au développement d'une application
de traitement d'image avec une architecture de type client-serveur.
Afin de permettre aux groupes d'approfondir un sujet qui les intéresse plus
particulièrement, on propose la structure suivante.

\begin{description}
\item[Noyau commun:] Chaque groupe devra réaliser l'implémentation du fonctionnement central de l'application client-serveur. Elle est décrite dans la section \verb!2 Noyau commun! et consiste principalement à articuler le contenu développé durant les premières semaines au sein des TP communs.

 Le code de cette partie fera l'objet d'un rendu intermédiaire.
  
\item [Extensions: ] Des suggestions d'extensions seront proposées en cours (répartition des charges entre client et serveur, amélioration de l'interface utilisateur, traitements d'image plus avancés,  généricité des algorithmes).
  
\end{description}


Chaque groupe peut faire évoluer ce document avec l'aval de son chargé de TD. Le cahier des besoins fera partie des rendus.\\


L'application devra permettre de traiter les images en niveau de gris et en
couleur enregistrées aux formats suivants :

\begin{itemize}
\item \verb!JPEG!
\item \verb!PNG!
\end{itemize}



\section{\label{sec:kernel}Noyau commun}

\subsection{Serveur}

\besoin{server:initImages}
{Initialiser un ensemble d'images présentes sur le serveur}
{
  Lorsque le serveur est lancé, il doit enregistrer toutes les images présentes à
  l'intérieur du dossier \verb!images!. Ce dossier \verb!images! doit exister à l'endroit où est lancé le serveur. Le serveur doit analyser l'arborescence à l'intérieur de ce dossier. Seuls les fichiers images correspondants aux formats d'image reconnus doivent être traités.
}
{
  Si le dossier \verb!images! n'existe pas depuis l'endroit où a été lancé le
  serveur, aucune image n'apparait, mais le client peut tout de meme upload ses images et utiliser le site.
}
{
  \begin{enumerate}
  \item Lancement de l'exécutable depuis un environnement vide possible, dans ce cas, la
    galerie sera vide.
  \item Mise en place d'un dossier de test nommé \verb!images! (:backend/src/main/resources) contenant au moins 2 niveaux de
    profondeur dans l'arborescence.
    Le dossier contiendra des documents avec des extensions non-reconnues comme
    étant des images (e.g. \verb!.txt!).
  \end{enumerate}
}

\besoin{server:manageImages}
{Gérer les images présentes sur le serveur}
{  
    Le serveur gère un ensemble d'images. Il stocke les données brutes de chaque image ainsi que les méta-données nécessaires aux réponses aux requêtes (identifiant, nom de fichier, taille de l'image, format,...). Le serveur peut :

    \begin{enumerate}
    \item accéder à une image via son identifiant,
    \item supprimer une image via son identifiant,
    \item ajouter une image,
    \item construire la liste des images disponibles (composée uniquement des métadonnées).
  \end{enumerate}
}
{}
{}

\besoin{server:processImage}
{Appliquer un algorithme de traitement d'image}
{
  Le serveur contient l'implémentation des algorithmes de traitement d'image proposés à l'utilisateur (voir partie \verb!2.4 Traitement d’images!).
}
{}
{}

\newpage

\subsection{Communication}

Pour l'ensemble des besoins, les codes d'erreurs à renvoyer sont précisés dans
le paragraphe "Gestion d'erreurs".\\


\besoin{comm:listImages}
{Transférer la liste des images existantes}
{
  La liste des images présentes sur le serveur doit être envoyée par le serveur lorsqu'il reçoit une requête \verb!GET! à l'adresse \verb!/images!.

  Le résultat sera fourni au format \verb!JSON!, sous la forme d'un tableau
  contenant pour chaque image un objet avec les informations suivantes :
  \begin{description}
  \item[Id:] L'identifiant auquel est accessible l'image (type \verb!long!)
  \item[Name:] Le nom du fichier qui a servi à construire l'image (type \verb!string!)
  \item[Type:] Le type de l'image (type \verb!org.springframework.http.MediaType!) 
  \item[Size:] Une description de la taille de l'image, par exemple \verb!640*480*3! pour
    une image en couleur de $640 \times 480$ pixels (type \verb!string!)
  \end{description}
}
{}
{
  Pour le dossier de tests spécifié dans \verb!Besoin 1!, la
  réponse attendue doit être comparée à la réponse reçue lors de l'exécution de la commande.
}

\besoin{comm:create}
{Ajout d'image}
{
  L'envoi d'une requête \verb!POST! à l'adresse \verb!/images! au serveur avec
  des données de type multimedia dans le corps doit ajouter une
  image à celles stockées sur le serveur (voir \verb!Besoin 2!).
}
{
  \begin{description}
  \item[201 Created:] La requête s'est bien exécutée et l'image est à présent
    sur le serveur.
  \item[415 Unsupported Media Type:] La requête a été refusée car le serveur ne
    supporte pas le format reçu (par exemple \verb!EXR!).
  \end{description}
}
{}

\besoin{comm:retrieve}
{Récupération d'images}
{
  L'envoi d'une requête \verb!GET! à une adresse de la forme \verb!/images/id!
  doit renvoyer l'image stockée sur le serveur avec l'identifiant \verb!id! (entier positif). En cas de succès, l'image est retournée dans le corps de la réponse.
}
{
  \begin{description}
  \item[200 OK:] L'image a bien été récupérée.
  \item[404 Not Found:] Aucune image existante avec l'identifiant \verb!id!.
  \end{description}
}
{}


\besoin{comm:delete}
{Suppression d'image}
{
  L'envoi d'une requête \verb!DELETE! à une adresse de la forme \verb!/images/id!
  doit effacer l'image stockée avec l'identifiant \verb!id! (entier positif).
}
{
  \begin{description}
  \item[200 OK:] L'image a bien été effacée.
  \item[404 Not Found:] Aucune image existante avec l'identifiant \verb!id!.
  \end{description}
}
{}

\besoin{comm:runAlgorithm}
{Exécution d'algorithmes par le serveur}
{
  L'envoi d'une requête \verb!GET! à une adresse de la forme
  \verb!/images/id?algorithm=X\&p1=Y! doit permettre de récupérer
  le résultat de l'exécution de l'algorithme \verb!X! avec le paramètre
  \verb!p1=Y!.
  Un exemple plus concret d'URL valide est:
  \verb!/images/23?algorithm=increaseLuminosity\&gain=25!

  En cas de succès, le serveur doit renvoyer l'image obtenue après traitement.
}
{
  \begin{description}
  \item[200 OK:] L'image a bien été traitée.
  \item[400 Bad Request:] Le traitement demandé n'a pas pu être validé par le
    serveur pour l'une des raisons suivantes:
    \begin{itemize}
    \item l'algorithme n'existe pas ;
    \item l'un des paramètres mentionné n'existe pas pour l'algorithme choisi ;
    \item la valeur du jeu de paramètres est invalide.
    \end{itemize}
    Le message d'erreur doit clarifier la source du problème.
  \item[404 Not Found:] Aucune image existante avec l'indice \verb!id!.
  \item[500 Internal Server Error:] L'exécution de l'algorithme a échoué pour
    une raison interne.
  \end{description}
}
{}

\newpage

\subsection{Client}
Les actions que peut effectuer l'utilisateur côté client induisent des requêtes envoyées au serveur. En cas d'échec d'une requête, le client doit afficher un message d'erreur explicatif.\\

\besoin{client:viewImages}
{Parcourir les images disponibles sur le serveur}
{
  L'utilisateur peut visualiser les images disponibles sur le serveur. La présentation visuelle prend la forme d'une galerie d'images, le client peut cliquer sur une image pour l'agrandir et ensuite parcourir la galerie via des flèches. Chaque image à une taille fixe (relativement à la page affichée). Une mise en évidence et un pointeur s'active au survol d'une image.
}
{}
{}

\besoin{client:selectImage}
{Sélectionner une image et lui appliquer un effet}
{
  L'utilisateur peut selectionner une image via le menu \verb!Home!. L'image est affichée sur la page. L'utilisateur peut choisir un des traitements d'image disponibles. Il peut être amené à préciser les paramètres nécessaires au traitement choisi (voir partie \verb!2.4 Traitement d’images!). L'image après traitement sera alors affichée dans la galerie, le client sera alors redirigé automatiquement sur la page galerie.
}
{}
{}


\besoin{client:saveImage}
{Enregistrer une image sur disque}
{
  L'utilisateur peut sauvegarder dans son système de fichier l'image chargée, avant ou après lui avoir appliqué un traitement. En cliquant sur une image dans la galerie puis sur l'icone de téléchargement.
}
{}
{}

\besoin{client:createImage}
{Ajouter une image aux images disponibles sur le serveur}
{
  L'utilisateur peut ajouter une image choisie dans son système de fichier aux images disponibles sur le serveur. Cet ajout n'est pas persistant au redémarrage du serveur (il n'y a pas d'ajout de fichiers côté serveur).
}
{}
{}



\besoin{client:delete}
{Suppression d'image}
{
  Le client peut choisir de supprimer une image préalablement sélectionnée. Elle n'apparaîtra plus dans les images disponibles sur le serveur. Cette suppression n'est pas persistante au redémarrage du serveur (il n'y a pas de suppression de fichiers côté serveur).
}
{}
{}

\newpage

\subsection{Traitement d'images}
\label{tai}

\besoin{tai:luminosity}
{Réglage de la luminosité}
{L'utilisateur peut augmenter ou diminuer la luminosité de l'image sélectionnée.}
{}
{}

\besoin{tai:equalizeHist}
{Égalisation d'histogramme}
{L'utilisateur peut appliquer une égalisation d'histogramme à l'image sélectionnée. L'égalisation sera apliquée sur le canal V de l'image représentée dans l'espace HSV.}
{}
{}

\besoin{tai:setHue}
{Filtre coloré}
{L'utilisateur peut choisir la teinte de tous les pixels de l'image sélectionnée de façon à obtenir un effet de filtre coloré.}
{}
{}

\besoin{tai:blur}
{Filtres de flou}
{L'utilisateur peut appliquer un flou à l'image sélectionnée. Il peut choisir entre le filtre moyenneur et gaussien dans la liste des algorithmes. Le filtre moyenneur doit être utilisé avec un paramètre qui correspond au niveau de flou que l'on souhaite (attention plus le niveau est grand plus cela mettra de temps à s'appliquer). En ce qui concerne le filtre gaussien, une valeur par défaut a été défini, il n'y a donc pas de paramètre.}
{}
{}

\besoin{tai:contour}
{Filtre de contour}
{L'utilisateur peut appliquer un détecteur de contour à l'image sélectionnée. Le résultat sera issu d'une convolution par le filtre de Sobel. La convolution sera appliquée sur la version en niveaux de gris de l'image.}
{}
{}

\newpage

\subsection{Besoins non-fonctionnels}

\besoin{bnf:serverCompatibility}
{Compatibilité du serveur}
{
  La partie serveur de l'application sera écrite en Java (JDK 11) avec les
  bibliothèques suivantes:
  \begin{itemize}
  \item \verb!org.springframework.boot! : version 2.6.0
  \item \verb!BoofCV! : version 0.39.1
  \end{itemize}

  Son fonctionnement devra être éprouvé sur au moins un des environnement
  suivants :
  \begin{itemize}
  \item Windows 10
  \item Ubuntu 20.04
  \item Debian Buster
  \item MacOS 11
  \end{itemize}
}
{}
{}

\besoin{bnf:clientCompatibility}
{Compatibilité du client}
{
  Le client sera écrit en JavaScript et s'appuiera sur la version \verb!3.x! du
  framework \verb!Vue.js!.

  Le client devra être testé sur au moins l'un des navigateurs Webs suivants,
  la version à utiliser n'étant pas imposée :
  \begin{itemize}
  \item Safari
  \item Google Chrome
  \item Firefox
  \end{itemize}
}
{}
{}

\besoin{bnf:documentation}
{Documentation d'installation et de test}
{
  La racine du projet devra contenir un fichier \verb!README.md! indiquant au
  moins les informations suivantes:
  \begin{itemize}
  \item Système(s) d'exploitation sur lesquels votre serveur a été testé, voir
    \verb!Besoin 19!.
  \item Navigateur(s) web sur lesquels votre client a été testé incluant la
    version de celui-ci, voir \verb!Besoin 20!.
  \end{itemize}
  
}
{}
{}

\subsection{Besoins additionnels}

\besoin{bnf:negative}
{Filtre négatif}
{
  L'utilisateur peut appliquer un filtre négatif à l'image sélectionnée. 
  Une image négative est une image dont les couleurs ont été inversées par rapport à l'originale.
}
{}
{}

\besoin{bnf:sideGray}
{Filtre SideGray}
{
  L'utilisateur peut appliquer un filtre "sideGray" à l'image sélectionnée. 
  Ce filtre permet de griser la moitié de l'image selectionnée par l'utilisateur (gauche ou droite selon le paramètre choisi) tandis que l'autre moitié
  reste en couleur.
}
{}
{}

\besoin{bnf:keepColor}
{Filtre KeepColor}
{
  L'utilisateur peut appliquer un filtre "KeepColor" à l'image sélectionnée. 
  Ce filtre permet de garder uniquement les pixels voulus de l'image tandis que le reste de l'image sera grisée.
  Voici les différentes couleurs qui peuvent être selectionnée pour être gardé sur l'image:
  \begin{itemize}
  \item Vert
  \item Bleu
  \item Rouge
  \item Jaune
  \item Violet
  \item Orange
  \item les couleurs chaudes (rouge, orange, jaune)
  \item les couleurs froides (vert, bleu, violet)
  \end{itemize}
}
{}
{}

\besoin{bnf:Solarize}
{Filtre Solarize}
{
  L'utilisateur peut appliquer un filtre "Solarize" à l'image sélectionnée. 
  Ce filtre permet d'inverser la tonalité de l'image.
}
{}
{}

\besoin{bnf:interfaceUtilisateur}
{Amélioration de l’interface utilisateur}
{
  L'interface utilisateur a été complétement modifié pour être plus pratique d'utilisation et plus ergonomique.
  Nous pouvons nous déplacer de la galerie vers le menu Home et vice-versa (le menu Home est défini par défaut).
  Les différentes modifcations effectuées sont explicitées dans les besoins ci-dessous.
}
{}
{}

\besoin{bnf:panelLeft}
{Panneau de gauche}
{
  Le panneau de gauche permet de selectionner les différentes images ainsi que les différents algorithmes :
  \begin{itemize}
  \item Sur le panneau des images (affiché par défaut), toutes les images sont affichées (format réduit) pour être pré-visualisées avant
  de cliquer dessus pour les afficher en plus grande. En dessous des images, nous pouvons choisir d'importer nos propres images via un bouton
  upload. L'importation d'une image se fait directement et la modification est appliquée en temps réel.
  \item Sur le panneau des algorithmes, tous les noms des algorithmes sont affichés. Nous pouvons cliquer dessus ce qui a pour effet de selectionner
  l'algorithme voulu. Les algorithmes nécessitant des paramètres "entier" se déroule pour afficher une barre "slide" pour selectionné la valeur.
  Pour les autres, ce sont des cases à sélectionner.
  Nous ne pouvons appliquer l'algorithme que si l'image est selectionnée et si on se trouve dans le menu des algorithmes.
  \end{itemize}
  }
{}
{}

\besoin{bnf:panelright}
{Panneau de droite}
{
  Le panneau de droite permet d'afficher l'image lorsqu'elle a été selectionnée dans le menu des images de gauche.
}
{}
{}

\besoin{bnf:panelBottom}
{Panneau du bas de la page}
{
  Le panneau du bas de la page permet de supprimer l'image choisie, de la télécharger ou d'appliquer un algorithme.
  La suppression d'une image se fait directement et la modification est appliquée en temps réel.
}
{}
{}

\besoin{bnf:loading}
{Chargement d'un algorithme}
{
  Lors de l'application d'un algorithme sur une image, un écran de chargement apparait sur la page.
}
{}
{}

\end{document}
